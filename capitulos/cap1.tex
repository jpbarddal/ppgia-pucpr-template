\chapter{Introdução}

Abaixo temos uma dissertação ou tese escrita em \LaTeX.
Com \LaTeX você não precisa ter pesadelos para fazer referências. Veja como é simples: \cite{knuth1984tex}.
Sim, já está no formato da ABNT e você não perdeu o sono por conta disso.

\lipsum[2-4]

\begin{longcitation}
	A boa notícia para os profissionais de software é que a economia mundial depende cada vez mais de software. Os sistemas que utilizam software intensivamente, que a tecnologia torna possível e a sociedade demanda, estão aumentando em tamanho, complexidade, distribuição e importância. A notícia ruim é que a expansão desses sistemas em tamanho, complexidade, distribuição e importância, empurram os limites do que nós, na indústria de software, sabemos como desenvolver. (...) Para aumentar o problema, o negócio continua a demandar produtividade crescente e melhor qualidade, com desenvolvimento e entrega mais rápidos. Adicionalmente, ainda, a disponibilidade de pessoal de desenvolvimento qualificado não é condizente com a demanda. (p. 3, tradução nossa)
\end{longcitation}

\lipsum[2-4]
\lipsum[2-4]

No Algoritmo \ref{alg:cap} temos um exemplo de um algoritmo.

\begin{algorithm}
\caption{Um algoritmo}\label{alg:cap}
\begin{algorithmic}[1]
\Require $n \geq 0$
\Ensure $y = x^n$
\State $y \gets 1$
\State $X \gets x$
\State $N \gets n$
\While{$N \neq 0$}
\If{$N$ is even}
    \State $X \gets X \times X$
    \State $N \gets \frac{N}{2}$  \Comment{Um comentário}
\ElsIf{$N$ is odd}
    \State $y \gets y \times X$
    \State $N \gets N - 1$
\EndIf
\EndWhile
\end{algorithmic}
\end{algorithm}

\section{Outra Seção}
\lipsum[2-4]

Agora um exemplo de nota de rodapé\footnote{Um rodapé.}.

\lipsum[2-4]

Abaixo temos também um exemplo de código-fonte (Código \ref{cod:fibonacci}), que é diferente de um algoritmo.
\begin{lstlisting}[language=Python, caption=A função de Fibonacci escrita em Python., label=cod:fibonacci]
	def fibonacci(n):
		if n <= 1:
			return n
		else:
			return fibonacci(n-1) + fibonacci(n-2)
\end{lstlisting}
	
	
\lipsum[2-4]		

\subsubsection{Uma subsubseção de exemplo}

Um exemplo de citação da Figura \ref{fig:logopucpr}.

\begin{figure}[!t]
	\centering
	\resizebox{0.5\textwidth}{!}{
		\includegraphics{template/PUCPR_logo.jpg}
	}
	\caption{A logo da PUCPR.}
	\label{fig:logopucpr}
\end{figure}	

\lipsum[2-4]	

\subsubsection{Outra subsubseção W}
	
\lipsum[2-4]	

\lipsum[2-4]		

\paragraph{Um parágrafo}

\lipsum[2-4]	
\lipsum[2-4]	
\paragraph{Mais um parágrafo}
\lipsum[2-4]	
\lipsum[2-4]	

Abaixo temos um exemplo de Quadro \ref{quad:quadro}.
\begin{quadro}
\centering
\caption{Quadro tal e tal.} \label{quad:quadro}
\begin{tabular}{c|c}
\hline
Coluna 1 & Coluna 2 \\
\hline
123 & 456\\
123 & 456\\
\hline
\end{tabular}

\end{quadro}

\lipsum[2-4]	
\lipsum[2-4]	


Agora um exemplo da Tabela \ref{tab:tabela}.

\begin{table}
	\centering
	\caption{Uma tabela com dados.}
	\label{tab:tabela}
	\begin{tabular}{|c|c|}
		\hline
		Coluna 1 & Coluna 2 \\
		\hline
		12 & 34 \\
		56 & 78 \\
		\hline	
	\end{tabular}
\end{table}

\lipsum[2-4]	
\lipsum[2-4]	

